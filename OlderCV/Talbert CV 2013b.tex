% LaTeX Curriculum Vitae Template
%
% Copyright (C) 2004-2009 Jason Blevins <jrblevin@sdf.lonestar.org>
% http://jblevins.org/projects/cv-template/
%
% You may use use this document as a template to create your own CV
% and you may redistribute the source code freely. No attribution is
% required in any resulting documents. I do ask that you please leave
% this notice and the above URL in the source code if you choose to
% redistribute this file.

\documentclass[letterpaper]{article}

\usepackage{hyperref}
\usepackage{geometry}
\usepackage{enumitem}
\usepackage{amssymb, amsmath, amsfonts}

% Comment the following lines to use the default Computer Modern font
% instead of the Palatino font provided by the mathpazo package.
% Remove the 'osf' bit if you don't like the old style figures.
\usepackage[T1]{fontenc}
\usepackage[sc,osf]{mathpazo}

% Set your name here
\def\name{Robert N. Talbert}

% Replace this with a link to your CV if you like, or set it empty
% (as in \def\footerlink{}) to remove the link in the footer:
\def\footerlink{http://jblevins.org/projects/cv-template/}

% The following metadata will show up in the PDF properties
\hypersetup{
  colorlinks = true,
  urlcolor = black,
  pdfauthor = {\name},
  pdfkeywords = {economics, statistics, mathematics},
  pdftitle = {\name: Curriculum Vitae},
  pdfsubject = {Curriculum Vitae},
  pdfpagemode = UseNone
}

\geometry{
  body={6.5in, 8.5in},
  left=1.0in,
  top=1.25in
}

% Customize page headers
\pagestyle{myheadings}
\markright{\name}
\thispagestyle{empty}

% Custom section fonts
\usepackage{sectsty}
\sectionfont{\rmfamily\mdseries\Large}
\subsectionfont{\rmfamily\mdseries\itshape\large}

% Other possible font commands include:
% \ttfamily for teletype,
% \sffamily for sans serif,
% \bfseries for bold,
% \scshape for small caps,
% \normalsize, \large, \Large, \LARGE sizes.

% Don't indent paragraphs.
\setlength\parindent{0em}

% Make lists without bullets
\renewenvironment{itemize}{
  \begin{list}{}{
    \setlength{\leftmargin}{1.5em}
	\setlength{\itemsep}{0in}
  }
}{
  \end{list}
}

\begin{document}

% Place name at left
{\Large \name}

% Alternatively, print name centered and bold:
%\centerline{\huge \bf \name}

\vspace{0.25in}

\begin{minipage}{0.45\linewidth}
  \href{http://www.gvsu.edu/}{Grand Valley State University} \\
  Department of Mathematics \\
  A-2-168 Mackinac Hall \\
  Allendale, MI 49401
\end{minipage}
\begin{minipage}{0.45\linewidth}
  \begin{tabular}{ll}
    Phone: & 616.331.8968 \\
    Email: & \href{mailto:talbertr@gvsu.edu}{\tt talbertr@gvsu.edu} \\
	Twitter: & \href{http://www.twitter.com/RobertTalbert}{@RobertTalbert} \\
    Homepage: & \href{http://faculty.gvsu.edu/talbertr/}{\tt http://faculty.gvsu.edu/talbertr/} \\
  \end{tabular}
\end{minipage}


% \section*{Personal}
% 
% \begin{itemize}
% \item Born on September 29, 1895.
% \item United States Citizen.
% \end{itemize}


\section*{Areas of Interest}

Cryptography; computational algebra; category theory; discrete mathematics for computer science; scholarship of teaching and learning including peer instruction, screencasting, the inverted classroom, and the use of technology to support active learning environments.


\section*{Education}

\begin{itemize}
  \item Ph.D. Mathematics, Vanderbilt University, 1997. Dissertation: \emph{Stratified and equivariant homology via homotopy colimits}, advised by Efstratios Prassidis. Areas of emphasis: Algebraic and geometric topology, category theory, ring theory, lattice theory, universal algebra.
	% \begin{description}
	% 	\item[Dissertation:] 
	% 	\item[Advisor:] 
	% 	\item[Areas:] 
	% \end{description}


  \item M.S. Mathematics, Vanderbilt University, 1994. Qualifying paper: \emph{The Leray-Serre spectral sequence, equivariance, and cohomology}. 

  \item B.S. Mathematics (\emph{magna cum laude in cursu honorum}), Tennessee Technological University, 1992. 

\end{itemize}


\section*{Employment}

\begin{itemize}
\item Associate Professor of Mathematics, Grand Valley State University 2011--present. 	
\item Assistant/Associate Professor of Mathematics and Computing Science, Franklin College 2001--2011. 
\item Assistant Professor of Mathematics, Bethel College (Indiana) 1997--2001. 
\item Master Teaching Fellow, Vanderbilt University Center for Teaching 1996--1997. 
\item Adjunct Faculty, Nashville State Community College 1995--1996. 
\item Graduate Teaching Scholar, Vanderbilt University 1993--1997. 
\end{itemize}


\section*{Teaching}

\subsection*{Courses taught or scheduled at GVSU}
MTH 201 (Calculus); MTH 202 (Calclulus 2); MTH 203 (Calculus 3); MTH 210 (Communicating in Mathematics); MTH 225 (Discrete Structures for Computer Science 1); MTH 227 (Linear Algebra 1); MTH 310 (Modern Algebra); MTH 312 (Cryptography and Privacy); MTH 325 (Discrete Structures for Computer Science 2); MTH 410 (Modern Algebra 2). 
 

\subsection*{Directed student projects at GVSU}
\begin{itemize}
	\item Elliptic curve cryptography (Christopher Grow, Winter 2012 for MTH 499).
	\item Fixed points and 2-cycles in columnar transposition ciphers (Beth Bjorkman, Summer 2012 for McNair Scholars Program and Fall 2012 for MTH 496). 
\end{itemize}


\subsection*{Courses taught prior to GVSU}

\begin{itemize}
	\item \textbf{Franklin College}: Quantitative Reasoning; Functions and Models; Calculus; Calculus II; Calculus III; Differential Equations; Linear Algebra; Methods of Problem Solving; Topics in Geometry; Modern Algebra; Special Topics (Cryptology); Computer Tools for Problem Solving; Operations Research; Cryptology, Privacy and Leadership; The Life and Works of C.S. Lewis.  
	\item \textbf{Bethel College}: College Algebra; Applied Calculus; Calculus I; Calculus II; Calculus III; Differential Equations; Modern Geometry; Abstract Algebra I; Abstract Algebra II; Topology; Christianity and the Life of the Mind. 
	\item \textbf{Vanderbilt University}: Calculus for Business and Life Science; Calculus I and II (liberal arts track); Calculus I and II (Science and Engineering track); Calculus III (Science and Engineering track). 
	\item \textbf{Nashville State Community College}: Algebra; Elementary Statistics; Business Mathematics. 
\end{itemize}


\subsection*{Directed student projects prior to GVSU}

Risk management and forecasting; Hyperbolic geometry and trigonometry; Elliptic curve cryptography; Computer investigations in geometry; Technology law; Geometry in art, architecture, and nature; Introduction to number theory; Mathematical methods in artificial intelligence; Taxicab geometry; Finite fields and applications; Computer modeling of hyperbolic geometry. 


\section*{Scholarship}

\subsection*{Research in progress}
\begin{itemize}
	\item Analysis of cycle structure of columnar transposition ciphers with $n \geq 3$ columns. (With Beth Bjorkman; \textbf{Collaboration with undergraduate student})
	\item The inverted classroom model in transition-to-proof courses and calculus. 
	\item Investigation into adoption and non-adoption of research-based teaching practices among STEM faculty at Grand Valley State University (With Scott Grissom, Shannon Biros, and Shaily Menon)
	\item Analysis of student motivations and strategies for learning in MTH 210: Communicating in Mathematics, in an inverted classroom setting. 
	% \item Investigation of effectiveness of free-response and graphical questions versus multiple choice questions in a ``bring-your-own-device'' implementation of classroom response systems in linear algebra. 
	\item Article in preparation: Talbert, R. (2013) Inverting the transition-to-proof course. 
	\item Article in preparation: Bjorkman, B. and Talbert, R. (2013) Fixed points of columnar transposition ciphers.
\end{itemize}


\subsection*{Peer-reviewed articles}

\begin{itemize}
% \item Talbert, R. (2013) Inverting the transition-to-proof classroom. Submitted to \emph{PRIMUS: Problems, Resources, and Issues in Undergraduate Mathematical Studies}. 
\item Talbert, R. (2013) Inverting the linear algebra classroom. To appear in \emph{PRIMUS: Problems, Resources, and Issues in Undergraduate Mathematical Studies}. 
\item Talbert, R. (2013) Learning MATLAB in the inverted classroom. \emph{Computers in Education Journal}, \textbf{4}(2): 89--100.  
\item Talbert, R. (2006) The cycle structure and order of the rail fence cipher. \emph{Cryptologia}, \textbf{30}(2):159--172. 
\item Talbert, R. (1999) An isomorphism between Bredon and Quinn homology via homotopy colimits. \emph{Forum Mathematicum}, \textbf{11}:591--616. 
\end{itemize}

\subsection*{Other articles}
\begin{itemize}
	\item Talbert, R. (2012) The inverted classroom. \emph{Colleagues}, Vol. 9: Iss. 1, Article 7. \url{http:// scholarworks.gvsu.edu/colleagues/vol9/iss1/7/}.  
	\item Talbert, R. (2011) Using MATLAB to teach problem-solving techniques to first-year liberal arts students.\emph{ Mathworks News and Notes}, Fall issue.
	\item Gash, J. and Talbert, R. (2011) Integrating spreadsheets, visualization tools, and computational knowledge engines in a liberal arts calculus course. \emph{Proceedings of the Twenty-Second International Conference on Technology in Collegiate Mathematics}: \url{http://archives.math.utk.edu/ICTCM/i/22/C004.html}. 
	\item Talbert, R. (2011) Teaching MATLAB to a non-canonical audience. \emph{Proceedings of the Twenty-Second International Conference on Technology in Collegiate Mathematics}: \url{http://archives.math.utk.edu/ICTCM/i/22/C006.html}. 
	\item Talbert, R. (2009) A tale of two wikis: Upper-level mathematics meets Web 2.0.\emph{Proceedings of the Twentieth International Conference on Technology in Collegiate Mathematics}: \url{http://archives.math.utk.edu/ICTCM/i/20/C009.html}.
\end{itemize}

\subsection*{Book reviews}
\begin{itemize}
	\item \emph{Virtual Reality and Animation for MATLAB and Simulink Users: Visualization of Dynamic Models and Control Simulations} for INFORMS Journal on Computing: \url{http://www.informs.org/Pubs/IJOC/Book-Reviews/Volume-24-2012}. 
	\item \emph{Elliptic Curves: Number Theory and Cryptography} for MAA Reviews, \url{http://tinyurl.com/4cr5r3}, 22 August 2008.
	\item \emph{Finite Fields and Applications} for MAA Reviews, \url{http://tinyurl.com/4w5eqt}, 21 May 2008.
\end{itemize}

\subsection*{Online publication}

Author of \textbf{Casting Out Nines} blog (\url{http://chronicle.com/blognetwork/castingoutnines}), a sponsored online publication of the \emph{Chronicle of Higher Education} on mathematics, teaching, and technology.


\subsection*{Invited talks, workshops, and panel participation}
\begin{itemize}
	\item Plenary speaker and instructor-in-residence, Appalachian College Association Teaching and Learning Institute, Hickory, NC June 2014. 
	\item Facilitator, Workshop on the inverted classroom. Delta College, University Center, MI February 2014. 
	\item Panelist, ``Assessment in Non-Traditional Classrooms''. Joint Meetings of the AMS/MAA, Baltimore, MD January 2014. 
	\item Facilitator, roundtable discussion on the flipped classroom, GVSU Faculty Teaching Roundtable, November 2013. 
	\item ``Deconstructing columnar transposition ciphers''. Mathematics Colloquium, Hope College, Holland, MI November 2013. 
	\item ``Deconstructing columnar transposition ciphers''. Mathematics Colloquium, Andrews University, Berrien Springs, MI October 2013.
	\item ``The inverted classroom and peer instruction: Designing classes for meaningful learning experiences.'' Keynote address, Michigan Mathematical Association of Two-Year Colleges, Auburn Hills, MI October 2013. 
	\item ``Giving your class an inverted classroom makeover''. Workshop for Michigan Mathematical Association of Two-Year Colleges, Auburn Hills, MI October 2013. 
	\item ``Flipping the college classroom: Transforming students into active learners''. Webinar for the Higher Ed Hero network, October 2013. 
	\item ``Better learning through voting''. Workshop at Ferris State University, Big Rapids, MI August 2013.
	\item ``Teaching and learning in the inverted classroom''. Workshop for faculty retreat at Lindsey Wilson College, Columbia, KY August 2013. 
	\item ``Teaching human beings''. Plenary address, Appalachian College Association Teaching and Learning Institute, Ferrum, VA June 2013. 
	\item ``Inverting the classroom to improve student learning and engagement''. Workshop given at at Appalachian College Association Teaching and Learning Institute, Ferrum, VA June 2013. 
	\item Participant in panel discussion on inquiry-based learning, Indiana MAA Section meeting, Indianapolis, IN October 2012. 
	\item ``Flip Your College Classroom: Increase Engagement and Experiential Learning''. Webinar given through Higher Ed Hero network, October 2012.
	\item ``Finding your next job''. Panel discussion presentation on Issues for Early Career Mathematicians, Mathematical Association of America MathFest, Madison, WI August 2012.
	\item ``Five questions about columnar transposition ciphers''. Guest presentation to Mathematics REU participants, Grand Valley State University, June 2012. 
	\item ``Flipping the Classroom: Overturning the Traditional Lecture''. Webinar given with Ike Shipley for The Blended Librarian, May 2012.
	\item ``Flipping the College Classroom.'' Webinar sponsored by Cengage Learning, March 2012.
	\item ``Flipping the College Classroom.'' Webinar given through American Mathematical Association of Two-Year Colleges, September 2011.
	\item  ``Inverting the Classroom, Improving Student Learning''. Presentation to graduate student teaching seminar, Mathematics Department, Indiana University Purdue University Indianapolis, Indianapolis, IN, March 2011. 
	\item ``Inquiry-Based MATLAB for General First-Year Students''. International Conference on Technology in Collegiate Mathematics, Denver, CO, March 2011. \url{http://bit.ly/g9ob42}
	\item ``Using Web-Based Presentation Tools to Promote Visual Thinking''. Workshop given at the Center for Research on Learning and Teaching, University of Michigan, Ann Arbor, MI, January 2011. \url{http://bit.ly/eI081T}
	\item ``Deconstructing Columnar Transposition Ciphers''. Department of Mathematical Sciences Faculty Colloquium; Ball State University, Muncie, IN, April 2009. \url{http://www.slideshare.net/rtalbert/deconstructing-columnar-transposition-ciphers}
	\item ``The Digital Signature Algorithm''. Guest lecture to MATH 390: Cryptography; Benedictine University, Lisle, IL, April 2008. 
	\item ``Protecting Ourselves with Mathematics: An Overview of Cryptology''. PBenedictine University Math Club; Benedictine University, Lisle, IL, April 2008.
\end{itemize}

\subsection*{Contributed talks}

\begin{itemize}
	\item ``Peer instruction in linear algebra''. Session on Innovative and Effective Ways to Teach Linear Algebra, Joint Meetings of AMS/MAA, Baltimore, MD January 2014. 
	\item ``A different kind of math: Addressing student difficulties with proof by flipping the transition-to-proof course''. Session on Flipping the Classroom, Joint Meetings of AMS/MAA, Baltimore, MD January 2014.
	\item ``Technology as a tool for self-regulated learning in an inverted calculus class''. Session on Teaching With Technology: Impact, Evaluation, and Reflection, Joint Meetings of AMS/MAA, Baltimore, MD January 2014.
	\item ``Inverting the transition-to-proof course''. Session on Research on the Teaching and Learning of Undergraduate Mathematics, Joint Meetings of AMS/MAA, Baltimore, MD January 2014.
	\item ``Clickers without the clickers: Using the web and students' personal devices for classroom response.'' GVSU Fall Teaching and Learning Conference, Grand Rapids, MI August 2013. 
	\item ``Making student learning visible in real time with peer instruction and web-based classroom response systems''. GVSU Scholarship of Teaching and Learning Academy, Grand Rapids, MI May 2013. 
	\item ``Transitioning to proofs in the inverted classroom''. Mathematical Association of America, Michigan Section meeting, Sault Ste. Marie, MI May 2013. 
	\item ``Using iPads to enhance the teaching and learning of mathematics''. GVSU Math in Action, Allendale, MI February 2013. 
	\item ``Learning MATLAB in the inverted classroom.'' American Society for Engineering Education Annual Conference, San Antonio, TX June 2012. 
	\item ``Classroom response systems in mathematics: Learning math better through voting''. GVSU Math in Action, Allendale, MI February 2012. 
	\item ``Making proofs click: Classroom response systems in transition-to-proof courses''. American Mathematical Society/Mathematical Association of America Joint Meetings, Boston, MA January 2012. 
	\item ``So you created a screencast. Now what?'' Techsmith, Inc. ScreencastCamp 2011, Okemos, MI August 2011. 
	\item ``Inquiry-based MATLAB for general first-year students''. International Conference on Technology in Collegiate Mathematics, Denver, CO March 2011. 
	\item ``Inverting the linear algebra classroom''. American Mathematical Society/Mathematical Association of America Joint Meetings, New Orleans, LA January 2011.
	\item ``A brief fly-through of cryptology for first-year students using active learning and common technology''.   American Mathematical Society/Mathematical Association of America Joint Meetings, New Orleans, LA January 2011.
	\item ``Teaching MATLAB to a Non-Canonical Audience''. International Conference on Technology in Collegiate Mathematics, Chicago, IL March 2010. 
	\item ``Integrating Spreadsheets, Visualization Tools, and Computational Knowledge Engines in a Liberal Arts Calculus Course'' (with J. Gash). International Conference on Technology in Collegiate Mathematics, Chicago, IL March 2010.
	\item ``A Tale of Two Wikis: Upper-Level Mathematics Courses meet Web 2.0''. International Conference on Technology in Collegiate Mathematics, San Antonio, TX March 2008.
\end{itemize}

\begin{center}
  \begin{footnotesize}
    List of contributed talks prior to 2008 available upon request
  \end{footnotesize}
\end{center}

\subsection*{Books}
\begin{itemize}
	\item Talbert, R. (2007) \emph{Test Bank} to accompany \emph{A Mathematical View of Our World}. Thomson Higher Education, Belmont, CA 2007.
\end{itemize}

\subsection*{Online course materials}
\begin{itemize}
	\item Screencasts for MTH 201: Calculus. \url{http://bit.ly/GVSUCalculus}. Created collaboratively in August-November 2013 with Prof. Marcia Frobish. 
	\item Screencasts for MTH 210: Communicating in Mathematics. \url{http://www.youtube.com/playlist?list=PL2419488168AE7001}. Created July-November 2012. 
\end{itemize}


\subsection*{Grants}
\begin{itemize}
	\item Co-Principal Investigator (with Scott Grissom, Shannon Biros, and Shaily Menon), National Science Foundation WIDER grant DUE-1256384, ``EAGER: GVSU Inventory of Instructional Practices'' September 2012, \$137,893.  
	\item GVSU Pew Technology Enhancement Grant 13-250, ``Implementing a Bring-Your-Own-Device Classroom Response System in Linear Algebra'' October 2012, \$2510. 
	\item GVSU Center for Scholarly and Creative Excellence, Faculty Scholarly Dissemination Grant-in-Aid, January 2012, \$500. 
\end{itemize}






\section*{Service}

\subsection*{Regional and national service activities}
\begin{itemize}
	\item Chair, American Society for Engineering Education Mathematics Division 2013--2014. 
	\item Program Chair and Chair-elect, American Society for Engineering Education Mathematics Division, 2012--2013. 
	\item Director of Mathematical Association of America Project NExT, Michigan Section 2013--present. 
	\item Editorial review board member of \emph{Mathematics Exchange} journal 2010--present. 
\end{itemize}


\subsection*{GVSU service activities}
\begin{itemize}
	\item Leader of Faculty Learning Community on the Inverted Classroom, 2013--2014. 
	\item Department Web Administrator, Mathematics Department 2012--present.  
	\item Instructional Resources Coordinator, Mathematics Department 2011--present. 
	\item Social Media coordinator, Mathematics Department 2011--present.
	\item Seminar Coordinator, Mathematics Department 2011--2012; Co-Coordinator 2013--2014. 
	\item New faculty mentor, Mathematics Department, 2013--2014. 
	\item Election Committee (\emph{ad hoc} to coordinate special election to fill Assistant Chair position), Mathematics Department Winter 2013. 
	\item Curriculum Committee, Mathematics Department 2012--2013. 
\end{itemize}

\subsection*{Service activities prior to GVSU}
\begin{itemize}
	\item Director, Dual-Degree Program in Engineering, Franklin College 2006--2011. 
	\item Promotion and Tenure Committee, Franklin College 2006--2011 (Chair, 2008--2009). 
	\item Mentor, High School Dual-Enrollment Programs, Franklin College 2007--2010. 
	\item Curricular Assessment and Planning Committee, Franklin College 2002--2006 (Chair, 2003--2006). 
	\item Administrative Committee, Bethel College 1999--2001. 
	\item Strategic Planning Committee, Bethel College 1998--2001. 
	\item Director, Honors Program, Bethel College 1998--2001. 
	\item Financial Aid Committee, Bethel College 1997--1998. 
\end{itemize}


\section*{Professional Development}

\subsection*{Coursework}
\begin{itemize}
	\item CS 101: Introduction to Computer Science (Online course offered by Udacity; completion with certificate April 2012). 
	\item Securing Digital Democracy (Online course offered by the University of Michigan through Coursera; completion with certificate December 2012).
	% \item Computing for Data Analysis (Online course offered by Johns Hopkins University through Coursera; completed with certificate January 2013).
	% \item CS 215: Algorithms (Online course offered by Udacity; expected completion in 2013). 
	% \item Introduction to Algorithms (Online course offered by Stanford University through Coursera; expected completion in March 2013).
	\item Cryptography I (Online course offered by Stanford University through Coursera; completion with certificate August 2013). 
	% \item Coding the Matrix: Linear Algebra through Computer Science Applications (Online course offered by Brown University through Coursera; completion with certificate August 2013). 
\end{itemize}


\subsection*{Workshops and minicourses}

\begin{itemize}
	\item The mathematics of paper folding. Mathematical Association of America MathFest, Madison, WI August 2012. 
	\item Discrete and computational geometry. American Mathematical Society/Mathematical Association of America Joint Meetings, Boston, MA January 2012. 
	\item Getting started in engineering education research. American Society for Engineering Education minicourse, Louisville, KY, June 2010. 
\end{itemize}

\subsection*{New course development}
\begin{itemize}
	\item \textbf{MTH 312: Cryptography and Privacy. }First offering at GVSU in Winter 2014 as part of Information, Innovation, and Technology theme in General Education curriculum. 
\end{itemize}


\section*{Professional Memberships}
\begin{itemize}
	\item International Association for Cryptologic Research 2006--2008 and 2013--present. 
	\item American Society for Engineering Education 2010--present.
	\item Association for Computing Machinery special interest group in Computer Science Education (ACM-SIGCSE), 2012--present.
	\item Mathematical Association of America 1997--present. 
	\item Mathematical Association of America, Michigan section 2011--present. 
\end{itemize}




\bigskip

% Footer
\begin{center}
  \begin{footnotesize}
    Last updated: \today \\
    %\href{\footerlink}{\texttt{\footerlink}}
  \end{footnotesize}
\end{center}

\end{document}