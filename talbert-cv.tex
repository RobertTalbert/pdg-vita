% Robert Talbert CV
% Last updated on 26-06-2014

\documentclass[letterpaper]{article}

\usepackage{hyperref}
\usepackage{geometry}
\usepackage{enumitem}
\usepackage{amssymb, amsmath, amsfonts}
\usepackage[T1]{fontenc}
\usepackage[sc,osf]{mathpazo}
\usepackage{bibentry}
% \usepackage{fontawesome}

\def\name{Robert N. Talbert}

% Replace this with a link to your CV if you like, or set it empty
% (as in \def\footerlink{}) to remove the link in the footer:
\def\footerlink{}

% The following metadata will show up in the PDF properties
\hypersetup{
  colorlinks = true,
  urlcolor = black,
  pdfauthor = {\name},
  pdfkeywords = {mathematics, mathematics education, computer science},
  pdftitle = {\name: Curriculum Vitae},
  pdfsubject = {Curriculum Vitae},
  pdfpagemode = UseNone
}

\geometry{
  body={6.5in, 8.5in},
  left=1.0in,
  top=1.25in
}

% Customize page headers
\pagestyle{myheadings}
\markright{\name}
\thispagestyle{empty}

% Custom section fonts
\usepackage{sectsty}
\sectionfont{\rmfamily\mdseries\Large}
\subsectionfont{\rmfamily\mdseries\itshape\large}

\setlength\parindent{0em}

% Make lists without bullets
\renewenvironment{itemize}{
  \begin{list}{}{
    \setlength{\leftmargin}{1.5em}
	\setlength{\itemsep}{0in}
  }
}{
  \end{list}
}

\begin{document}

% Place name at left
{\Large \name}


\vspace{0.25in}

\begin{minipage}{0.45\linewidth}
  \href{http://www.gvsu.edu/}{Grand Valley State University} \\
  Department of Mathematics \\
  A-2-168 Mackinac Hall \\
  Allendale, MI 49401 \\
  616.331.8968
\end{minipage}
\begin{minipage}{0.45\linewidth}
    \href{mailto:talbertr@gvsu.edu}{talbertr@gvsu.edu} \\
    \href{http://faculty.gvsu.edu/talbertr/}{rtalbert.org} \\
    \href{http://www.github.com}{github.com/RobertTalbert} \\
    \href{http://www.twitter.com/RobertTalbert}{twitter.com/RobertTalbert} \\
\end{minipage}


%   \begin{tabular}{ll}
%     \href{mailto:talbertr@gvsu.edu}{talbertr@gvsu.edu} \\
%     \href{http://faculty.gvsu.edu/talbertr/}{rtalbert.org} \\
%     \href{http://www.github.com}{github.com/RobertTalbert} \\
%     \href{http://www.twitter.com/RobertTalbert}{twitter.com/RobertTalbert} \\
%     % \href{http://www.google.com/+RobertTalbert}{google.com/+RobertTalbert}
%  %    Email: & \href{mailto:talbertr@gvsu.edu}{\tt talbertr@gvsu.edu} \\
% 	% Twitter: & \href{http://www.twitter.com/RobertTalbert}{twitter.com/RobertTalbert} \\
%  %    Homepage: & \href{http://faculty.gvsu.edu/talbertr/}{\tt http://rtalbert.org} \\
%   \end{tabular}
% \end{minipage}


\section*{Areas of Interest}

\begin{itemize}
  Scholarship of teaching and learning in undergraduate mathematics (flipped learning, integrating technology, self-regulated learning, and alternative grading practices); applications of algebra to functional programming and computer science.
\end{itemize}

\section*{Education}

\begin{itemize}
  \item Ph.D. Mathematics, Vanderbilt University, 1997. Dissertation: \emph{Stratified and equivariant homology via homotopy colimits}, advised by Efstratios Prassidis. Areas of emphasis: Algebraic and geometric topology, category theory, ring theory, lattice theory, universal algebra.


  \item M.S. Mathematics, Vanderbilt University, 1994. Qualifying paper: \emph{The Leray-Serre spectral sequence, equivariance, and cohomology}.

  \item B.S. Mathematics (\emph{magna cum laude in cursu honorum}), Tennessee Technological University, 1992.

\end{itemize}

\section*{Employment}

\begin{itemize}
\item Professor of Mathematics, Grand Valley State Univerisity, 2017--present. 
\item Associate Professor of Mathematics, Grand Valley State University, 2011--2017.
\item Associate Professor of Mathematics and Computing Science, Franklin College, 2007--2011.
\item Assistant Professor of Mathematics and Computing Science, Franklin College, 2001--2007.
\item Assistant Professor of Mathematics, Bethel College (Indiana), 1997--2001.
\item Master Teaching Fellow, Vanderbilt University Center for Teaching, 1996--1997.
\item Adjunct Faculty, Nashville State Community College, 1995--1996.
\item Graduate Teaching Scholar, Vanderbilt University, 1993--1997.
\end{itemize}


\section*{Awards}
\begin{itemize}
	\item Pew Teaching with Technology Award, Grand Valley State University, 2015.
	\item Distinguished Educator Award, American Society for Engineering Education Mathematics Division, 2015.
	\item GVSU nominee for Michigan Distinguished Professor of the Year, 2015.
\end{itemize}


\section*{Teaching experience}

\subsection*{Courses taught or scheduled at GVSU}
MTH 124 (Functions and Models); MTH 201 (Calculus); MTH 202 (Calclulus 2); MTH 203 (Calculus 3); MTH 210 (Communicating in Mathematics); MTH 225 (Discrete Structures for Computer Science 1); MTH 227 (Linear Algebra 1);  MTH 312 (Cryptography and Privacy); MTH 325 (Discrete Structures for Computer Science 2); MTH 350 (Modern Algebra); MTH 450 (Modern Algebra 2).


\subsection*{Directed student projects at GVSU}
Elliptic curve cryptography (Christopher Grow, Winter 2012 for MTH 499); Fixed points and 2-cycles in columnar transposition ciphers (Beth Bjorkman, Summer 2012 for McNair Scholars Program and Fall 2012 for MTH 496).

\subsection*{Courses taught prior to GVSU}

\begin{itemize}
	\item \textbf{Franklin College}: Quantitative Reasoning; Functions and Models; Calculus; Calculus II; Calculus III; Differential Equations; Linear Algebra; Methods of Problem Solving; Topics in Geometry; Modern Algebra; Special Topics (Cryptology); Computer Tools for Problem Solving; Operations Research; Cryptology, Privacy and Leadership; The Life and Works of C.S. Lewis.
	\item \textbf{Bethel College}: College Algebra; Applied Calculus; Calculus I; Calculus II; Calculus III; Differential Equations; Modern Geometry; Abstract Algebra I; Abstract Algebra II; Topology; Christianity and the Life of the Mind.
	\item \textbf{Vanderbilt University}: Calculus for Business and Life Science; Calculus I and II (liberal arts track); Calculus I and II (Science and Engineering track); Calculus III (Science and Engineering track).
	\item \textbf{Nashville State Community College}: Algebra; Elementary Statistics; Business Mathematics.
\end{itemize}


\subsection*{Directed student projects prior to GVSU}

Risk management and forecasting; Hyperbolic geometry and trigonometry; Elliptic curve cryptography; Computer investigations in geometry; Technology law; Geometry in art, architecture, and nature; Introduction to number theory; Mathematical methods in artificial intelligence; Taxicab geometry; Finite fields and applications; Computer modeling of hyperbolic geometry.


\section*{Scholarship}

\subsection*{Scholarship in progress}
\begin{itemize}
%	\item Co-author, \textit{Instructional Practices Guide} forthcoming from Mathematical Association of America.
%	\item Effects of the use of student-created video content on self-regulated learning behaviors in calculus (with Shelly Smith; in preparation for the \textit{International Journal on Innovations in Online Education} and two conference presentations).
	\item Experiences of students with learning disabilities in flipped learning environments (with Amy Schelling, College of Education).
	\item Effects of flipped instruction on self-regulated learning skills and behaviors of university Calculus students (with Marcia Frobish, Department of Mathematics)
	\item ``A place for learning: A review of the scholarly literature on active learning classrooms'' (with Anat Mor-Avi, Illinois Institute of Technology). 
\end{itemize}


\subsection*{Book}

\begin{itemize}
  \item Talbert, R. \textit{Flipped Learning: A Guide for Higher Education Faculty}. Sterling, VA: Stylus Publications, 2017. ISBN  978-1-62036-432-1
\end{itemize}

\subsection*{Peer-reviewed publications}

\begin{itemize}
	\item Morena, M., Smith, S., and Talbert, R. (2018) Video made the calculus star.  \textit{PRIMUS: Problems, Resources, and Issues in Mathematics Undergraduate Studies},  DOI: 10.1080/10511970.2017.1396568
	\item LaRose, P. and Talbert, R. (2016). Editorial for Teaching with Technology Special Issue of PRIMUS. \textit{PRIMUS: Problems, Resources, and Issues in Mathematics Undergraduate Studies}, 26(6):505--508.
  \item Talbert, R. (2016) Four Strategies for Effective Assessment in a Flipped Learning Environment. In \textit{Flipping the College Classroom: Practical Advice from Faculty.} (pp. 75-- 77). Madison, WI: Magna Publications.
  \item Talbert, R. (2016) Three critical conversations started and sustained by flipped learning. In \textit{Flipping the College Classroom: Practical Advice from Faculty}. (pp. 32--35). Madison, WI: Magna Publications.
  \item Talbert, R. (2016) Flipped calculus: A gateway to lifelong learning in mathematics. In J. Waldrop and M. Bowdon (Ed.), \textit{Best Practices for Flipping the College Classroom}. (pp. 29--43). New York, NY: Routledge Press.
	\item Talbert, R. (2016) Flipped calculus: A gateway to lifelong learning in mathematics. In J. Waldrop and M. Bowdon (Eds.) \textit{Best Practices for Flipping the College Classroom} (pp. 29--43). New York, NY: Routledge Press.
	\item Talbert, R. (2015). Inverting the transition-to-proof classroom. \textit{PRIMUS: Problems, Resources, and Issues in Mathematics Undergraduate Studies}, 25(8):614--626.
	\item Bjorkman, B. and Talbert, R. (2015). Fixed Points of Columnar Transpositions. \textit{Journal of Discrete Mathematical Sciences and Cryptography}, 18(5):541--557.
	\item Talbert, R. (2014). Inverting the Linear Algebra Classroom. \textit{PRIMUS: Problems, Resources, and Issues in Mathematics Undergraduate Studies}, 24(5):361--374.
	\item Talbert, R. (2014). The inverted classroom in introductory calculus: Best practices and potential benefits for the preparation of engineers. In \textit{Proceedings of the American Society for Engineering Education 2014 Annual Conference and Exposition}. American Society for Engineering Education. \textbf{Winner, Best Paper Award, ASEE Mathematics Division.}
	\item Talbert, R. (2012). Inverted classroom. Colleagues, 9(1, Article 7):1--2.
	\item Talbert, R. (2012). Learning MATLAB in the inverted classroom. In \textit{Proceedings of the American Society for Engineering Education 2012 Annual Conference and Exposition}. American Society for Engineering Education.
	\item Talbert, R. (2011). Teaching MATLAB to a Non-canonical audience. In \textit{Electronic Proceedings of the Twenty-second Annual International Conference on Technology in Collegiate Mathematics}, pages 1--8.
	\item Gash, J. and Talbert, R. (2011). Integrating spreadsheets, visualization tools, and computational knowledge engines in a liberal arts calculus course. In \textit{Electronic Proceedings of the Twenty-second Annual International Conference on Technology in Collegiate Mathematics}, pages 1--6.
	\item Talbert, R. (2009). A Tale of two wikis: Upper-level mathematics meets Web 2.0. In \textit{Electronic Proceedings of the Twentieth Annual International Conference on Technology in Collegiate Mathematics}, pages 1–-5.
	\item Talbert, R. (2007)  \emph{Test Bank} to accompany \emph{A Mathematical View of Our World}. Thomson Higher Education, Belmont, CA.
	\item Talbert, R. (2006). The Cycle Structure and Order of the Rail Fence Cipher. \textit{Cryptologia}, 30(2), 159-172.
	\item Talbert, R. (1999). An Isomorphism between Bredon and Quinn Homology. \textit{Forum Mathematicum}, 11(5).
\end{itemize}




% \bibliography{TalbertReferences.bib}{}
% \bibliographystyle{apa}
% \begin{itemize}
% 	\item \bibentry{GavinLaRose2016}
% 	\item \textsc{Talbert, R.}
% 	\item \bibentry{Talbert2015}
% 	\item \bibentry{Talbert2014}
% 	\item \bibentry{Talbert2014-ASEE} \textbf{Winner, Best Paper Award} in the ASEE Mathematics Division.
% 	\item \bibentry{Talbert2012}
% 	\item \bibentry{talbert2012learning}
% 	\item \bibentry{Talbert2011}
% 	\item \bibentry{TalbertGash2011}
% 	\item \bibentry{Talbert2009}
% 	\item \textsc{Talbert, R.}  \emph{Test Bank} to accompany \emph{A Mathematical View of Our World}. Thomson Higher Education, Belmont, CA (2007).
% 	\item \bibentry{Talbert2006}
% 	\item \bibentry{Talbert1999}
% \end{itemize}


% \item Talbert, R. (2015) ``Flipped Calculus: A Gateway to Lifelong Learning in Mathematics''.  In \emph{Best Practices for Flipping the College Classroom}, J. Waldrop and M. Bowdon, editors, Routledge Press, 29--43.
% \item Bjorkman, B. and Talbert, R. (2015) Fixed points of columnar transpositions. \emph{Journal of Discrete Mathematical Sciences and Cryptography} 18(5):541--557.
% \item Talbert, R. (2015) Inverting the transition-to-proof course. \emph{PRIMUS: Problems, Resources, and Issues in Undergraduate Mathematical Studies} 25(8):614--626.
% \item Talbert, R. (2014) The inverted classroom in introductory calculus: Best practices and potential benefits for the preparation of engineers. \emph{Proceedings of the American Society for Engineering Education 2014 Annual Conference}.  \textbf{Winner of Best Paper Award 2014, ASEE Mathematics Division.}
% \item Talbert, R. (2013) Inverting the linear algebra classroom. Inverting the linear algebra classroom. \emph{PRIMUS: Problems, Resources, and Issues in Mathematics Undergraduate Studies}, 24:5, 361-374, \\ DOI: 10.1080/10511970.2014.883457
% \item Talbert, R. (2013) Learning MATLAB in the inverted classroom. \emph{Computers in Education Journal}, \textbf{4}(2): 89--100.
% \item Talbert, R. (2006) The cycle structure and order of the rail fence cipher. \emph{Cryptologia}, \textbf{30}(2):159--172.
% \item Talbert, R. (1999) An isomorphism between Bredon and Quinn homology via homotopy colimits. \emph{Forum Mathematicum}, \textbf{11}:591--616.
% \end{itemize}

\subsection*{Other publications}
\begin{itemize}
	\item Talbert, R. (2015) ``Four Assessment Strategies for the Flipped Learning Environment''. \textit{Faculty Focus}, 10 August 2015, \url{http://bit.ly/1OtJ8Zi}.
	\item Talbert, R. (2015) ``Three Critical Conversations Started and Sustained by Flipped learning''. \textit{Faculty Focus}, 2 March 2015, \url{http://bit.ly/298OSxU}.
	% \item Talbert, R. (2012) The inverted classroom. \emph{Colleagues}, Vol. 9: Iss. 1, Article 7. \url{http:// scholarworks.gvsu.edu/colleagues/vol9/iss1/7/}.
	\item Talbert, R. (2011) Using MATLAB to teach problem-solving techniques to first-year liberal arts students.\emph{ Mathworks News and Notes}, Fall issue.
	% \item Gash, J. and Talbert, R. (2011) Integrating spreadsheets, visualization tools, and computational knowledge engines in a liberal arts calculus course. \emph{Proceedings of the Twenty-Second International Conference on Technology in Collegiate Mathematics}: \url{http://archives.math.utk.edu/ICTCM/i/22/C004.html}.
	% \item Talbert, R. (2011) Teaching MATLAB to a non-canonical audience. \emph{Proceedings of the Twenty-Second International Conference on Technology in Collegiate Mathematics}: \url{http://archives.math.utk.edu/ICTCM/i/22/C006.html}.
	% \item Talbert, R. (2009) A tale of two wikis: Upper-level mathematics meets Web 2.0.\emph{Proceedings of the Twentieth International Conference on Technology in Collegiate Mathematics}: \url{http://archives.math.utk.edu/ICTCM/i/20/C009.html}.
\end{itemize}

\subsection*{Book reviews}
\begin{itemize}
	\item \emph{Integrating Educational Technology into Teaching} by M.D. Roblyer, for Pearson, Inc., December 2015.
	\item \emph{Virtual Reality and Animation for MATLAB and Simulink Users: Visualization of Dynamic Models and Control Simulations} for INFORMS Journal on Computing: \url{http://www.informs.org/Pubs/IJOC/Book-Reviews/Volume-24-2012}.
	\item \emph{Elliptic Curves: Number Theory and Cryptography} for MAA Reviews, \url{http://tinyurl.com/4cr5r3}, 22 August 2008.
	\item \emph{Finite Fields and Applications} for MAA Reviews, \url{http://tinyurl.com/4w5eqt}, 21 May 2008.
\end{itemize}

\subsection*{Online publications}

\begin{itemize}
	\item Author of blog (\url{http://rtalbert.org/}) on mathematics, teaching, and technology. The blog was a sponsored publication of the \emph{Chronicle of Higher Education} until August 2015 (\url{http://chronicle.com/blognetwork/castingoutnines}).
\end{itemize}


\subsection*{Invited talks, workshops, and panel participation}
\begin{itemize}
   \item ``World building: Your role as a contributor to workplace culture''. Plenary address to Project NExT, Denver, CO, August 2018. 
   \item ``Flipped learning in mathematics: Principles, research and practice''. Plenary address to State University System of Florida Math Action Planning Conference, Orlando, FL, May 2018. 
   \item ``Unbounded spaces: Designing learning spaces with the rights of the learner in mind''. Keynote, 14th Annual Teaching and Learning Conference, University of British Columbia -- Okanogan, Okanogan, BC, Canada, May 2018. 
   \item ``Flipped learning in higher education''. Invited address to Steelcase, Inc. Los Angeles office, Los Angeles, CA, April 2018. 
   \item ``Humanizing Computation''. TEDxGVSU, Grand Rapids, MI, March 2018. 
   \item ``Flipped learning: Roots, Research, and Future Directions''. Invited address, Indiana University, Bloomington, IN, February 2018. 
   \item ``Four future challenges for flipped learning''. Invited talk and panel participation, EDUCAUSE Learning Initiative conference, New Orleans, LA, January 2018. 
   \item ``Time, space, and activity: Redesigning college courses for flipped learning''. Invited session, Merging Minds and Technology conference, Boston, MA, November 2017. 
   \item ``Flipped learning: A gateway to learning for life''. Keynote, XII International LASLAB Symposium, Vitoria-Gasteiz, Spain, May 2017. 
  \item ``Creating active learning environments in university mathematics courses with flipped learning'', workshop for mathematics faculty at St. Michael's College and MAA Northeastern Section, Colchester, VT, February 2017.
  \item ``Designing courses for significant flipped learning experiences'', workshop for faculty at St. Michael's College, Colchester, VT, February 2017.
  	\item ``Flipped learning: A gateway to lifelong learning'', workshop at Washington and Lee University, Lexington, VA, August 2016.
	\item ``Flipped learning: A gateway to lifelong learning'', webinar to faculty at California State Polytechnic Institute, August 2016.
	\item ``Flipped learning: A gateway to the self-regulated life''. Workshop given at Teaching and Learning Symposium, California State Polytechnic Institute, Pomona, CA, June 2016. \url{http://bit.ly/CPP-workshop}.
	\item ``The Self-Regulated Life: Teaching for Students' Lifelong Pursuit of Knowledge and Fulfillment''. Invited keynote address for Teaching and Learning Symposium, California State Polytechnic Institute, Pomona, CA, June 2016. \url{http://rtalbert.org/calpolypomona}.
	\item ``Specifications Grading: Restructure Assessments for Your College Course''. Webinar through Higher Ed Hero, December 2015.
	\item ``Adventures in Online Calculus''. Invited talk to Valparaiso University Mathematics Department Colloquium, Valparaiso, IN December 2015. \url{http://rtalbert.org/adventures}
	\item ``Adventures in Online Calculus''. Invited talk to GVSU Mathematics Department seminar, Allendale, MI December 2015. \url{http://rtalbert.org/adventures}
	\item Facilitator, GVSU Faculty Teaching Roundtable on standards-based and specifications grading, Allendale, MI November 2015.
	\item Flipped learning workshop for Technology Enhanced Instruction Community of Practice, PATH Group, Bethlehem, PA October 2015.
	\item ``Twenty-First Century Technology for Twenty-First Century Learners''. UWI/Guardian Group Premium Teaching Open Lecture, Kingston, Jamaica October 2015. \url{http://rtalbert.org/uwi}
	\item ``Teaching with Technology: Reimagining the University Classroom for the 21st Century''. Workshop for faculty at the University of the West Indies, Kingston, Jamaica October 2015.
	\item ``Rethinking Class Time Using Accessible Technology''. Invited keynote, Kansas City Regional Mathematics Technology Expo, Kansas City, MO October 2015.
	\item ``Crafting a Sustainable Career through Better Teaching''. Invited address, Missouri Section Project NExT, Kansas City, MO October 2015.
	\item ``Creating Flipped Learning Experiences in the College Mathematics Classroom''. Two-day minicourse offered at MAA MathFest, Washington, DC August 2015.
	\item ``Assessment Strategies for Flipped Learning Experiences''. Webinar through Magna Publications, September 2015.
	\item ``Implementing and Assessing Flipped Learning in Face-to-Face and Online Contexts''. Invited keynote, Hybrid Learning Network, Holland, MI June 2015.
	\item ``An ABC for Effective Flipped Learning''. Invited keynote, Innovation Insights in Quantitative Business, Toronto, Canada May 2015.
	\item ``Self-Regulated Learning in the Calculus Classroom''. Invited address to Project ADVANCE cohort, Syracuse and New York City, NY May 2015.
	\item ``Best Practices in Flipped Learning Design'', Webinar for Magna Publications, March 2015.
	\item ``Exploring the flipped learning model''. Workshop given to faculty at Lenoir-Rhyne University, Hickory NC January 2015.
	\item ``Exploring the flipped learning model''. Workshop given to faculty at Wilfrid Laurier University, Waterloo ON December 2014. \url{http://roberttalbert.github.io/wlu}
	\item ``Students at the center: The why and the how of student-centered, inquiry-focused instruction.'' Keynote presentation to faculty at Lenoir-Rhyne University, Hickory NC January 2015. \url{http://roberttalbert.github.io/lenoirrhyne}
	\item Three videos on the role of lecture and construction of screencasts for \emph{An Introduction to Evidence-Based Undergraduate STEM Teaching}, massively open online course offered through Coursera, October--December 2014.
	\item ``(Re:)Designing Class for Flipped Learning Experiences''. Invited talk to faculty at California Polytechnic University, San Luis Obispo CA October 2014. \url{http://roberttalbert.github.io/calpoly}
	\item ``Flipping the college classroom''. Webinar given through Higher Ed Hero, October 2014.
	\item ``Formative pre-assessment in a flipped mathematics class using online tools''. Invited talk to faculty at Leeds University, Leeds, UK (online), September 2014. \url{http://roberttalbert.github.io/leeds}
	\item ``(Re:)Designing Class for Flipped Learning Experiences''. Webinar to faculty at Mount Aloysius College (PA), September 2014. \url{http://roberttalbert.github.io/mtaloysius}
	\item Workshop facilitator on the flipped classroom, GVSU Fall Teaching Conference, Grand Rapids, MI August 2014.
	\item ``Flipping the university mathematics classroom: A gateway to lifelong learning.'' Invited keynote address at Workshop on Innovations in University Mathematics Teaching, Cardiff University, Cardiff, Wales UK July 2014. \url{http://roberttalbert.github.io/cardiffuniv}
	\item ``The conditions for invention: Using educational technology to make us more human''. Plenary address, Appalachian College Association Teaching and Learning Institute, Hickory, NC June 2014.
	\item ``(Re:)Designing class for flipped learning experiences''.  Workshop given at at Appalachian College Association Teaching and Learning Institute, Hickory, NC June 2014.
	\item ``Four things I wish I had known about the flipped classroom''. Keynote address to faculty at Ecole Centrale Paris, Paris, France (online) June 2014.
	\item Plenary speaker and instructor-in-residence, Appalachian College Association Teaching and Learning Institute, Hickory, NC June 2014.
	\item Workshop on flipping the classroom for GVSU faculty, with Matt Roberts. Grand Valley State University, Grand Rapids, MI May 2014.
	\item Facilitator, Workshop on the inverted classroom. Delta College, University Center, MI February 2014.
	\item Panelist, ``Assessment in Non-Traditional Classrooms''. Joint Meetings of the AMS/MAA, Baltimore, MD January 2014.
	\item Facilitator, roundtable discussion on the flipped classroom, GVSU Faculty Teaching Roundtable, November 2013.
	\item ``Deconstructing columnar transposition ciphers''. Mathematics Colloquium, Hope College, Holland, MI November 2013.
	\item ``Deconstructing columnar transposition ciphers''. Mathematics Colloquium, Andrews University, Berrien Springs, MI October 2013.
	\item ``The inverted classroom and peer instruction: Designing classes for meaningful learning experiences.'' Keynote address, Michigan Mathematical Association of Two-Year Colleges, Auburn Hills, MI October 2013.
	\item ``Giving your class an inverted classroom makeover''. Workshop for Michigan Mathematical Association of Two-Year Colleges, Auburn Hills, MI October 2013.
	\item ``Flipping the college classroom: Transforming students into active learners''. Webinar for the Higher Ed Hero network, October 2013.
	\item ``Better learning through voting''. Workshop at Ferris State University, Big Rapids, MI August 2013.
	\item ``Teaching and learning in the inverted classroom''. Workshop for faculty retreat at Lindsey Wilson College, Columbia, KY August 2013.
	\item ``Teaching human beings''. Plenary address, Appalachian College Association Teaching and Learning Institute, Ferrum, VA June 2013.
	\item ``Inverting the classroom to improve student learning and engagement''. Workshop given at at Appalachian College Association Teaching and Learning Institute, Ferrum, VA June 2013.
	\item Participant in panel discussion on inquiry-based learning, Indiana MAA Section meeting, Indianapolis, IN October 2012.
	\item ``Flip Your College Classroom: Increase Engagement and Experiential Learning''. Webinar given through Higher Ed Hero network, October 2012.
	\item ``Finding your next job''. Panel discussion presentation on Issues for Early Career Mathematicians, Mathematical Association of America MathFest, Madison, WI August 2012.
	\item ``Five questions about columnar transposition ciphers''. Guest presentation to Mathematics REU participants, Grand Valley State University, June 2012.
	\item ``Flipping the Classroom: Overturning the Traditional Lecture''. Webinar given with Ike Shipley for The Blended Librarian, May 2012.
	\item ``Flipping the College Classroom.'' Webinar sponsored by Cengage Learning, March 2012.
	\item ``Flipping the College Classroom.'' Webinar given through American Mathematical Association of Two-Year Colleges, September 2011.
	\item  ``Inverting the Classroom, Improving Student Learning''. Presentation to graduate student teaching seminar, Mathematics Department, Indiana University Purdue University Indianapolis, Indianapolis, IN, March 2011.
	\item ``Inquiry-Based MATLAB for General First-Year Students''. International Conference on Technology in Collegiate Mathematics, Denver, CO, March 2011. \url{http://bit.ly/g9ob42}
	\item ``Using Web-Based Presentation Tools to Promote Visual Thinking''. Workshop given at the Center for Research on Learning and Teaching, University of Michigan, Ann Arbor, MI, January 2011. \url{http://bit.ly/eI081T}
	\item ``Deconstructing Columnar Transposition Ciphers''. Department of Mathematical Sciences Faculty Colloquium; Ball State University, Muncie, IN, April 2009. \url{http://www.slideshare.net/rtalbert/deconstructing-columnar-transposition-ciphers}
	\item ``The Digital Signature Algorithm''. Guest lecture to MATH 390: Cryptography; Benedictine University, Lisle, IL, April 2008.
	\item ``Protecting Ourselves with Mathematics: An Overview of Cryptology''. PBenedictine University Math Club; Benedictine University, Lisle, IL, April 2008.
\end{itemize}

\subsection*{Contributed talks}

\begin{itemize}
	\item ``Flipped learning with Jupyter: Experiences, best practices, and supporting research'' (with Lorena Barba). JupyterCon, New York City, NY, August 2018.
	\item ``Computational thinking in discrete mathematics using Python and Jupyter notebooks'', MAA Session on Discrete Mathematics in the Undergraduate Curriculum, AMS/MAA Joint Meetings, Atlanta, GA January 2017.
  \item ``Making Learning Visible with Student-Generated Video Content'' (with Shelly Smmith), MAA Session on Creation and Implementation of Effective Homework Assignments, AMS/MAA Joint Meetings, Atlanta, GA January 2017.
	\item ``Making Learning Visible with Student-Generated Video Content''. Teaching Professor Technology Conference, Atlanta, GA October 2016.
	\item ``Flipped Infrastructures for Inquiry-Based Learning''. Legacy of R.L. Moore/IBL Conference, Austin, TX June 2015.
	\item ``The inverted classroom in introductory calculus: Best practices and potential benefits for the preparation of engineers.'' Mathematics Division paper session, American Society for Engineering Education annual conference, Indianapolis, IN June 2014.
	\item ``Peer instruction in linear algebra''. Session on Innovative and Effective Ways to Teach Linear Algebra, Joint Meetings of AMS/MAA, Baltimore, MD January 2014.
	\item ``A different kind of math: Addressing student difficulties with proof by flipping the transition-to-proof course''. Session on Flipping the Classroom, Joint Meetings of AMS/MAA, Baltimore, MD January 2014.
	\item ``Technology as a tool for self-regulated learning in an inverted calculus class''. Session on Teaching With Technology: Impact, Evaluation, and Reflection, Joint Meetings of AMS/MAA, Baltimore, MD January 2014.
	\item ``Inverting the transition-to-proof course''. Session on Research on the Teaching and Learning of Undergraduate Mathematics, Joint Meetings of AMS/MAA, Baltimore, MD January 2014.
	\item ``Clickers without the clickers: Using the web and students' personal devices for classroom response.'' GVSU Fall Teaching and Learning Conference, Grand Rapids, MI August 2013.
	\item ``Making student learning visible in real time with peer instruction and web-based classroom response systems''. GVSU Scholarship of Teaching and Learning Academy, Grand Rapids, MI May 2013.
	\item ``Transitioning to proofs in the inverted classroom''. Mathematical Association of America, Michigan Section meeting, Sault Ste. Marie, MI May 2013.
	\item ``Using iPads to enhance the teaching and learning of mathematics''. GVSU Math in Action, Allendale, MI February 2013.
	\item ``Learning MATLAB in the inverted classroom.'' American Society for Engineering Education Annual Conference, San Antonio, TX June 2012.
	\item ``Classroom response systems in mathematics: Learning math better through voting''. GVSU Math in Action, Allendale, MI February 2012.
	\item ``Making proofs click: Classroom response systems in transition-to-proof courses''. American Mathematical Society/Mathematical Association of America Joint Meetings, Boston, MA January 2012.
	\item ``So you created a screencast. Now what?'' Techsmith, Inc. ScreencastCamp 2011, Okemos, MI August 2011.
	\item ``Inquiry-based MATLAB for general first-year students''. International Conference on Technology in Collegiate Mathematics, Denver, CO March 2011.
	\item ``Inverting the linear algebra classroom''. American Mathematical Society/Mathematical Association of America Joint Meetings, New Orleans, LA January 2011.
	\item ``A brief fly-through of cryptology for first-year students using active learning and common technology''.   American Mathematical Society/Mathematical Association of America Joint Meetings, New Orleans, LA January 2011.
	\item ``Teaching MATLAB to a Non-Canonical Audience''. International Conference on Technology in Collegiate Mathematics, Chicago, IL March 2010.
	\item ``Integrating Spreadsheets, Visualization Tools, and Computational Knowledge Engines in a Liberal Arts Calculus Course'' (with J. Gash). International Conference on Technology in Collegiate Mathematics, Chicago, IL March 2010.
	\item ``A Tale of Two Wikis: Upper-Level Mathematics Courses meet Web 2.0''. International Conference on Technology in Collegiate Mathematics, San Antonio, TX March 2008.
\end{itemize}

\begin{center}
  \begin{footnotesize}
    List of contributed talks prior to 2008 available upon request
  \end{footnotesize}
\end{center}


\subsection*{Online course materials}
\begin{itemize}
	\item Jupyter notebooks for MTH 325: Discrete Structures for Computer Science 2. Created during Winter 2016 semester. Repository at \url{https://github.com/RobertTalbert/discretecs/tree/master/lessons}.
	\item Screencasts for MTH 201: Calculus. \url{http://bit.ly/GVSUCalculus}. Created collaboratively in August-November 2013 with Prof. Marcia Frobish.
	\item Screencasts for MTH 202: Calculus 2 \url{http://bit.ly/GVSUCalculus2}. Coordinated creation of this playlist with several other members of the Mathematics Department.
	\item Screencasts for MTH 210: Communicating in Mathematics. \url{http://www.youtube.com/playlist?list=PL2419488168AE7001}. Created July-November 2012.
\end{itemize}


\subsection*{Grants}
\begin{itemize}
	\item GVSU Center for Scholarly and Creative Excellence, Collaborative Research Grant, April 2018, \$7500. 
	\item Michigan Association for Computer Users in Learning Grant, ``Technology for the Implementation of Peer Instruction and Interactive Engagement Pedagogies in Calculus and Discrete Structures'', June 2014, \$1200.
	\item GVSU Center for Scholarly and Creative Excellence, Dissemination Grant, May 2014, \$750.
	\item Co-Principal Investigator (with Scott Grissom, Shannon Biros, and Shaily Menon), National Science Foundation WIDER grant DUE-1256384, ``EAGER: GVSU Inventory of Instructional Practices'' September 2012, \$137,893.
	\item GVSU Pew Technology Enhancement Grant 13-250, ``Implementing a Bring-Your-Own-Device Classroom Response System in Linear Algebra'' October 2012, \$2510.
	\item GVSU Center for Scholarly and Creative Excellence, Faculty Scholarly Dissemination Grant-in-Aid, January 2012, \$500.
\end{itemize}


%%%%%


\section*{Service}

\subsection*{Regional, national, and international service activities}
\begin{itemize}
	\item Editorial board, \emph{Journal of Open Source Education}, 2017--present. 
	\item Flipped Learning Research Fellow, Flipped Learning Global Initiative, 2016--present.
	\item Working writing group, chapter in \textit{Instructional Practices Guide}, Mathematical Association of America, 2016--present.
	\item Chair, Social Media Taskforce, Mathematical Association of America, 2016.
	\item Editorial board, \emph{PRIMUS: Problems, Resources, and Issues in Mathematics Undergraduate Studies}, 2015--present.
	\item Associate editor, \emph{Fields Mathematics Education Journal}, 2014--present.
	\item Guest co-editor, Special Issue on Teaching with Technology, \emph{PRIMUS: Problems, Resources, and Issues in Mathematics Undergraduate Studies}, 2014.
	\item Chair, American Society for Engineering Education Mathematics Division 2013--2014.
	\item Program Chair and Chair-elect, American Society for Engineering Education Mathematics Division, 2012--2013.
	\item Director of Mathematical Association of America Project NExT, Michigan Section 2013--2014.
	\item Editorial review board member of \emph{Mathematics Exchange} journal 2010--present.
\end{itemize}

\subsection*{GVSU University-level service activities}

\begin{itemize}
  \item Faculty Associate, Robert and Mary Pew Faculty Teaching and Learning Center, 2017--present.
	\item Facilitator, Faculty Learning Community on Standards-Based and Specifications Grading, 2016--2017.
	\item Pew Faculty Teaching and Learning Center Advisory Council, 2014--2017; chair 2015--2017.
	\item University Academic Policy and Standards Committee, 2014--2016.
	\item Faculty advisor, National Society of Collegiate Scholars GVSU chapter, 2014--2015.
	\item Facilitator, Faculty Learning Community on the Inverted Classroom, 2013--2014.
\end{itemize}

\subsection*{GVSU Unit-level service activities}

% All items below are associated with the Mathematics Department.

\begin{itemize}
	\item Assistant Chair, Department of Mathematics, 2018--present. 
	\item Head of task force to study online and hybrid Mathematics courses, 2016--2017.
	\item Head of task force to study recruitment and retention of Mathematics majors, 2016--2017.
	\item Student Affairs Committee, 2014--present, chair 2015--2016.
	\item Instructional Resources Coordinator, 2011--present.
	\item MTH 124 Course Design Taskforce, 2015--2016.
	\item Social Media coordinator, 2011--2015.
	\item Department Web Administrator, 2012--2014.
	\item Election Committee (\emph{ad hoc} to coordinate special election to fill Assistant Chair position), Mathematics Department Winter 2013.
	\item New faculty mentor, 2013--2014.
	\item Curriculum Committee, 2012--2013.
	\item Seminar Coordinator, 2011--2012; Co-Coordinator 2013--2014.
\end{itemize}


\subsection*{Service activities prior to GVSU}
\begin{itemize}
	\item Director, Dual-Degree Program in Engineering, Franklin College 2006--2011.
	\item Promotion and Tenure Committee, Franklin College 2006--2011 (Chair, 2008--2009).
	\item Mentor, High School Dual-Enrollment Programs, Franklin College 2007--2010.
	\item Curricular Assessment and Planning Committee, Franklin College 2002--2006 (Chair, 2003--2006).
	\item Administrative Committee, Bethel College 1999--2001.
	\item Strategic Planning Committee, Bethel College 1998--2001.
	\item Director, Honors Program, Bethel College 1998--2001.
	\item Financial Aid Committee, Bethel College 1997--1998.
\end{itemize}


\section*{Professional Development}

\subsection*{Coursework}
\begin{itemize}
  \item Certification program in Research Methods for the Social Sciences. Consists of courses in Quantitative Methods, Qualititative Methods, Basic Statistics, Inferential Statistics, and Capstone Project. Online courses offered by Coursera; January--June 2017.
	\item Introduction to Functional Programming. Online course offered by edX; completed with certificate January 2015.
	\item The Data Scientist's Toolbox. Online course offered by Coursera; completed November 2014.
	\item Programming Foundations with Python. Online course offered by Udacity; completed September 2014.
	\item CS 215: Algorithms. Online course offered by Udacity; completed with certificate with highest distinction May 2014.
	\item History and Future of (Mostly) Higher Education. Online course offered by Duke University through Coursera; completed with certificate with distinction January 2014.
	\item Cryptography I. Online course offered by Stanford University through Coursera; completed with certificate August 2013.
	\item Securing Digital Democracy. Online course offered by the University of Michigan through Coursera; completed with certificate December 2012.
	\item CS 101: Introduction to Computer Science. Online course offered by Udacity; completed with certificate April 2012.
\end{itemize}


\subsection*{Workshops and minicourses}

\begin{itemize}
	\item The mathematics of paper folding. Mathematical Association of America MathFest, Madison, WI August 2012.
	\item Discrete and computational geometry. American Mathematical Society/Mathematical Association of America Joint Meetings, Boston, MA January 2012.
	\item Getting started in engineering education research. American Society for Engineering Education minicourse, Louisville, KY, June 2010.
\end{itemize}

\subsection*{New course development}
\begin{itemize}
	\item \textbf{Hybrid section of MTH 124: Functions and Models}, offered Fall 2019. 
	\item \textbf{Hybrid section of MTH 201: Calculus}, offered Fall 2018. 
	\item \textbf{Online section of MTH 201: Calculus}, offered in Spring/Summer 2015 and Spring/Summer 2016.
	\item \textbf{MTH 124, Functions and Models}. Served on task force to design and propose the course.
	\item \textbf{MTH 225 and MTH 325, Discrete Structures for Computer Science 1--2}. Developed Jupyter notebook-based online platform for textbook and student work, implemented Winter and Fall 2016.
	\item \textbf{MTH 312: Cryptography and Privacy. }First offering at GVSU in Winter 2014 as part of Information, Innovation, and Technology theme in General Education curriculum.
\end{itemize}


\section*{Professional Memberships}
\begin{itemize}
	\item American Society for Engineering Education 2010--2018.
	\item Association for Computing Machinery special interest group in Computer Science Education (ACM-SIGCSE), 2012--2018.
	\item Mathematical Association of America 1997--present.
	\item Mathematical Association of America, Michigan section 2011--present.
	\item Michigan Association for Computer Users in Learning, 2014--2016.
\end{itemize}


\bigskip

% Footer
\begin{center}
  \begin{footnotesize}
    Last updated: \today \\
    %\href{\footerlink}{\texttt{\footerlink}}
  \end{footnotesize}
\end{center}

\end{document}
